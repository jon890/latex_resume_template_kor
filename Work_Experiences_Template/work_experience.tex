%-------------------------
% Resume in Latex
% Original author : Sourabh Bajaj
% Adaptation : Hyunggi Chang (changh95)
% License : MIT
%------------------------

\documentclass[letterpaper,11pt]{article}

\usepackage{latexsym}
\usepackage{kotex} % 한글 사용 가능! 
\usepackage[empty]{fullpage}
\usepackage{titlesec}
\usepackage{marvosym}
\usepackage[usenames,dvipsnames]{color}
\usepackage{verbatim}
\usepackage{enumitem}
\usepackage[hidelinks]{hyperref}
\usepackage{fancyhdr}
\usepackage[english]{babel}
\usepackage{tabularx}
\usepackage{amsmath}

\pagestyle{fancy}
\fancyhf{} % clear all header and footer fields
\fancyfoot{}
\renewcommand{\headrulewidth}{0pt}
\renewcommand{\footrulewidth}{0pt}

% Adjust margins
\addtolength{\oddsidemargin}{-0.5in}
\addtolength{\evensidemargin}{-0.5in}
\addtolength{\textwidth}{1in}
\addtolength{\topmargin}{-0.5in}
\addtolength{\textheight}{1.0in}

\urlstyle{same}

\raggedbottom
\raggedright
\setlength{\tabcolsep}{0in}

% Sections formatting
\titleformat{\section}{
  \vspace{-4pt}\scshape\raggedright\large
}{}{0em}{}[\color{black}\titlerule \vspace{-2pt}]

%-------------------------
% Custom commands - Do Look into this area, if you wish to customise further!
% 포맷이 마음에 들지 않으시다면 이 부분을 수정하세요!

\newcommand{\resumeItem}[1]{
  \item\small{
    {#1 \vspace{-2pt}}
  }
}

\newcommand{\resumeProject}[3]{
  \vspace{0pt}\item
    \begin{tabular*}{0.97\textwidth}[t]{l@{\extracolsep{\fill}}r}
      #1 & \small #2 \\
      {#3}
    \end{tabular*}\vspace{-5pt}
}

\newcommand{\resumeSubProject}[2]{
  \item{
    {#1 \vspace{0pt}}
    {#2}
  }
}

\newcommand{\resumeSubheading}[4]{
  \vspace{-1pt}\item
    \begin{tabular*}{0.97\textwidth}[t]{l@{\extracolsep{\fill}}r}
      \textbf{#1} & #2 \\
      \textit{\small#3} & \textit{\small #4} \\
    \end{tabular*}\vspace{-5pt}
}

\newcommand{\resumeSubItem}[2]{\resumeItem{#1}{#2}\vspace{-4pt}}

\renewcommand{\labelitemii}{$\circ$}

\newcommand{\resumeProjectListStart}{\begin{itemize}[leftmargin=*]}
\newcommand{\resumeProjectListEnd}{\end{itemize}}

\newcommand{\resumeSubProjectListStart}{\begin{itemize}[leftmargin=*]}
\newcommand{\resumeSubProjectListEnd}{\end{itemize}}

\newcommand{\resumeSubHeadingListStart}{\begin{itemize}[leftmargin=*]}
\newcommand{\resumeSubHeadingListEnd}{\end{itemize}}
\newcommand{\resumeItemListStart}{\begin{itemize}}
\newcommand{\resumeItemListEnd}{\end{itemize}\vspace{-5pt}}

%-------------------------------------------
%%%%%%  CV STARTS HERE  %%%%%%%%%%%%%%%%%%%%%%%%%%%%


\begin{document}

%----------HEADING-----------------
\begin{tabular*}{\textwidth}{l@{\extracolsep{\fill}}r}
  \textbf{\href{https://github.com/jon890}{\Large 김병태 (경력 기술서)}} & Github: jon890 \\
  \href{https://github.com/jon890}{https://github.com/jon890} & Email : \href{mailto:jon89071@gmail.com}{jon89071@gmail.com} \\
  {} & Mobile : (+82) 010-2753-2647
\end{tabular*}

%-----------PROJECTS-----------------
\section{\textbf{1. 소셜카지노 게임 슬롯 서비스 개발 프로젝트}}
  \resumeProjectListStart
    \resumeProject
    {\textbf{소속}: (주) NHN} {2024.06 - 진행 중}
    
    \textbf{주요 성과}:
    
    본 프로젝트에서 진행한 내용으로는 \\
    1. 6개 슬롯 게임 백엔드 개발 및 슬롯 로직 템플릿화 \\ 
    2. 슬롯 API 구조 추상화 및 리팩토링 \\
    3. 시뮬레이터 성능 최적화 및 테스트 환경 구축 \\
    4. RTP 부양 시스템 개발}

    \textbf{사용한 기술}: Java, Spring Framework, Spring Cloud, MySQL, Redis, Kafka, AWS

    \resumeSubProjectListStart
        \resumeSubProject {\textbf{1.1 6개 슬롯 게임 백엔드 개발}}
        
        {
        서로 다른 메커니즘을 가진 6개 슬롯 게임의 게임 로직을 구현하는 것을 목표로 둠 \\
        각 게임의 로직을 구현하는 것은 큰 무리는 없었음 \\
        그 중에서도 클러스터링, 텀블링 게임을 주로 맡아서 개발했고, 이 외에도 페이라인, 웨이 방식의 모든 슬롯을 개발할 수 있음 \\
        기존의 로직을 파악하고, 코드를 최대한 추상화 및 공통화 할 수 있도록 꾸준히 개선을 시도했음 \\
        위와 같은 시도 끝에, 결국 페이 방식을 관리자가 쉽게 변경하면서 즉시 해당 페이로 변경했을 떄, 결과물은 어떻게 나오는지 판단할 수 있게 되었음

        \vspace{7mm}

        \resumeSubProject{\textbf{1.2 슬롯 API 구조화 및 리팩토링}}
        
        {
        기존 복사-붙여넣기 방식으로 개발된 다양한 스핀 API들을 체계적으로 구조화하는 것을 목표로 둠. 일반 스핀, 바이피처 스핀, 바이피처 티켓 스핀과 일반/튜토리얼/치트 플레이 모드들이 중복 코드로 개발되어 있던 문제를 전략 패턴을 활용하여 해결함. 구현한 기능들은 아래와 같음. 
        
        \vspace{2mm}
        
        \textbf{1.2.1. 스핀 타입별 전략 패턴 적용}
        
        해당 기능은 기존에 각각 다른 API로 구현되어 있던 일반 스핀, 바이피처 스핀, 바이피처 티켓 스핀을 하나의 통합된 구조로 개선하기 위해 개발되었음. 스핀 타입별로 다른 동작을 전략 객체로 분리하여 런타임에 적절한 전략을 선택할 수 있도록 설계함. 이를 통해 코드 중복을 80\% 이상 제거하고 새로운 스핀 타입 추가 시 확장성을 확보함. \\
        이를 통해, 바이피처 티켓 스핀이라는 신규 기능 대응에도 무리없이 대응할 수 있었음
        
        \vspace{2mm}
        
        \textbf{1.2.2. 플레이 모드별 동작 분리}
        
        해당 기능은 일반 플레이, 튜토리얼 플레이, 치트 플레이 모드에서 발생하는 서로 다른 로직을 체계적으로 분리하기 위해 개발되었음. 각 모드별로 다른 검증 규칙, 스핀 결과 생성, 로그 처리 등을 별도의 전략으로 구현하여 모드 간 간섭 없이 독립적으로 동작할 수 있도록 개선함. 특히 튜토리얼 모드의 특수한 요구사항들을 유연하게 처리할 수 있게 되었음. \\
        특히 튜토리얼에서 여러 스핀 결과를 시나리오로 만들어서 내려주는 기능이 신규로 필요하게 되었는데, 쉽게 대응할 수 있었음
        
        \vspace{7mm}

        }

        \resumeSubProject{\textbf{1.3 시뮬레이터 성능 최적화 및 테스트 환경 구축}}
        
       {
        시뮬레이터의 Out of Memory 문제를 해결하고 테스트 가능한 구조로 개선하는 것을 목표로 둠. 힙덤프 분석을 통한 메모리 최적화와 테스트 용이성을 위한 구조적 개선을 통해 안정적이고 검증 가능한 시뮬레이터 플랫폼을 구축함. 구현한 기능들은 아래와 같음. 
        
        \vspace{2mm}
        
        \textbf{1.3.1. OOM 문제 해결 및 메모리 최적화}
        
        해당 기능은 시뮬레이터에서 발생하는 Out of Memory 에러를 근본적으로 해결하기 위해 개발되었음. 힙덤프 분석 결과 게임 변동성 계산을 위해 모든 승리 금액을 List에 저장하여 long 타입 객체가 대량으로 생성되는 문제를 발견함. 기존의 전체 데이터 저장 후 분산 계산 방식을 온라인 분산 계산 알고리즘(Welford's Algorithm)으로 변경하여 메모리 사용량을 95\% 이상 감소시킴.
        
        \vspace{2mm}
        
        \textbf{1.3.2. 테스트 용이성을 위한 구조 개선}
        
        해당 기능은 기존에 테스트하기 어려운 구조로 되어 있던 시뮬레이터를 테스트 가능한 형태로 개선하기 위해 개발되었음. 스핀 파라미터가 매우 복잡하여 테스트 작성이 어려웠던 문제를 해결하기 위해 테스트용 파라미터를 자동 생성하는 추상 테스트 클래스를 작성함. 단순 유닛테스트로는 복잡한 모듈 테스트가 어려워 통합테스트 방식으로 전환하고, 게임 데이터를 메모리에 효율적으로 로드하는 JUnit Extension을 개발함.

        \vspace{2mm}
        
        \textbf{1.3.3. 슬롯 테스트 플랫폼 구축}

        해당 기능은 모든 슬롯 게임에서 공통으로 사용할 수 있는 테스트 플랫폼을 구축하기 위해 개발되었음. 추상 슬롯 테스트 클래스를 기반으로 추상 시뮬레이터 테스트 클래스를 계층적으로 설계하여 각 게임별 특성에 맞는 테스트를 효율적으로 작성할 수 있도록 함. 이를 통해 팀원들이 쉽게 테스트를 추가할 수 있는 환경을 조성하여 코드 품질과 신뢰성을 크게 향상시킴.
        
        \vspace{7mm}
        }
        
        \resumeSubProject{\textbf{1.4 RTP 부양 시스템 개발}}
        
        {
        사용자의 게임 경험 개선을 위한 RTP 부양 시스템을 개발하는 것을 목표로 둠 \\
        이를 위해, 좋은 스핀 결과를 지속적으로 공급해줄 수 있는 스레드를 작게 두고 지속적으로 캐쉬 스핀을 만들어 내고 관리하는 시스템을 개발함 \\
        해당 기능을 위해 모든 슬롯 게임 및 전반적인 슬롯 서버의 아키텍처를 분석하고 개선해야 했음 \\
        예를 들면, 잭팟의 경우, 가짜로 만들어내는 스핀은 잭팟에 대한 구현체를 아무런 동작하지 않는 NoOpJackpotService와 같이 추가적으로 대응해야할 부분이 있었음 \\
        이 개발도 위에 진행한 1.2 슬롯 API 구조화 및 리팩토링 덕분에 쉽게 RTP 부양에 대한 스핀 결과 제공을 쉽게 추가할 수 있었음 \\
        
        \vspace{4mm}
        }

    \resumeSubProjectListEnd

  \resumeProjectListEnd

%-------------------------------------------
\section{\textbf{2. 글로벌 승부 예측 서비스 개발 프로젝트}}
  \resumeProjectListStart
    \resumeProject
    {\textbf{소속}: (주) NHN} {2022.12 - 2024.05}
    
    \textbf{주요 성과}:
    
    본 프로젝트에서 진행한 내용으로는 \\
    1. 어드민 시스템 구축 \\ 
    2. KYC 서버 구축 \\
    3. 프론트엔드 빌드 시스템 개선 등이 있음}

    \textbf{사용한 기술}: Java, Spring Framework, MySQL, Redis, RabbitMQ, Azure
  \resumeProjectListEnd
\end{document}
