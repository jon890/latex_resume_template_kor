%-------------------------
% Resume in Latex
% Original Author : Sourabh Bajaj
% Adaptation : Hyunggi Chang
% License : MIT
%------------------------

\documentclass[letterpaper,11pt]{article}

\usepackage{latexsym}
\usepackage[empty]{fullpage}
\usepackage{titlesec}
\usepackage{marvosym}
\usepackage[usenames,dvipsnames]{color}
\usepackage{verbatim}
\usepackage{enumitem}
\usepackage[hidelinks]{hyperref}
\usepackage{fancyhdr}
\usepackage[english]{babel}
\usepackage{tabularx}
\usepackage{amsmath}
\usepackage{kotex} % Enable Korean!

\pagestyle{fancy}
\fancyhf{} % clear all header and footer fields
\fancyfoot{}
\renewcommand{\headrulewidth}{0pt}
\renewcommand{\footrulewidth}{0pt}

% Adjust margins
\addtolength{\oddsidemargin}{-0.5in}
\addtolength{\evensidemargin}{-0.5in}
\addtolength{\textwidth}{1in}
\addtolength{\topmargin}{-0.5in}
\addtolength{\textheight}{1.0in}

\urlstyle{same}

\raggedbottom
\raggedright
\setlength{\tabcolsep}{0in}

% Sections formatting
\titleformat{\section}{
  \vspace{-4pt}\scshape\raggedright\large
}{}{0em}{}[\color{black}\titlerule \vspace{-2pt}]

%-------------------------
% Custom commands
\newcommand{\resumeItem}[1]{
  \item\small{
    {#1 \vspace{-2pt}}
  }
}

\newcommand{\resumeSummary}[1]{
  \item
    \begin{tabular*}{0.97\textwidth}[t]{l@{\extracolsep{\fill}}r}
      #1
    \end{tabular*}
}

\newcommand{\resumeSubheading}[4]{
  \vspace{-1pt}\item
    \begin{tabular*}{0.97\textwidth}[t]{l@{\extracolsep{\fill}}r}
      \textbf{#1} & #2 \\
      \textit{\small#3} & \textit{\small #4} \\
    \end{tabular*}\vspace{-5pt}
}

\newcommand{\resumeEmployment}[4]{
  \vspace{-1pt}\item
    \begin{tabular*}{0.97\textwidth}[t]{l@{\extracolsep{\fill}}r}
      \textbf{#1} & #2 \\
      \textit{\small#3} & \textit{\small #4} \\
    \end{tabular*}\vspace{-5pt}
}

\newcommand{\resumeProject}[2]{
  \vspace{-1pt}\item
    \begin{tabular*}{0.97\textwidth}[t]{l@{\extracolsep{\fill}}r}
      \textbf{#1} \\
      \small{#2} \\
    \end{tabular*}\vspace{-5pt}
}

\newcommand{\resumeResearch}[5]{
  \vspace{-1pt}\item
    \begin{tabular*}{0.97\textwidth}[t]{l@{\extracolsep{\fill}}r}
      \textbf{#1} & #2 \\
      \textit{\small#3} {\small #4 \vspace{-2pt}} & \textit{\small #5} \\
    \end{tabular*}\vspace{-5pt}
}

\newcommand{\resumeTalk}[2]{
  \vspace{-1pt}\item
    \begin{tabular*}{0.97\textwidth}[t]{l@{\extracolsep{\fill}}r}
      \textbf{#1} & #2 \\
    \end{tabular*}\vspace{-5pt}
}

\newcommand{\resumeSkills}[1]{
  \item
    \begin{tabular*}{0.97\textwidth}[t]{l@{\extracolsep{\fill}}r}
      #1
    \end{tabular*}
}

\newcommand{\resumeCommunity}[3]{
  \vspace{-1pt}\item
    \begin{tabular*}{0.97\textwidth}[t]{l@{\extracolsep{\fill}}r}
      \textbf{#1} & #2 \\
      \textit{\small#3} \\
    \end{tabular*}\vspace{-5pt}
}

\newcommand{\resumeSubItem}[2]{\resumeItem{#1}{#2}\vspace{-4pt}}

\renewcommand{\labelitemii}{$\circ$}

\newcommand{\resumeSubHeadingListStart}{\begin{itemize}[leftmargin=*]}
\newcommand{\resumeSubHeadingListEnd}{\end{itemize}}

\newcommand{\resumeEmploymentListStart}{\begin{itemize}[leftmargin=*]}
\newcommand{\resumeEmploymentListEnd}{\end{itemize}}

\newcommand{\resumeItemListStart}{\begin{itemize}}
\newcommand{\resumeItemListEnd}{\end{itemize}\vspace{-5pt}}

%-------------------------------------------
%%%%%%  CV STARTS HERE  %%%%%%%%%%%%%%%%%%%%%%%%%%%%


\begin{document}

%----------HEADING-----------------
\begin{tabular*}{\textwidth}{l@{\extracolsep{\fill}}r}
  \textbf{\href{http://github.com/jon890}{\Large 김병태}} & Github: jon890  \\
  \href{http://github.com/jon890}{http://github.com/jon890} & Email : \href{mailto:jon89071@gmail.com}{jon89071@gmail.com} \\
  {} & Mobile : (+82) 010-2753-2647
\end{tabular*}

%-----------Summary-----------------
\section{Summary}
    \resumeSummary{
    백엔드 개발 및 AI를 활용한 개발 효율화에 관심이 많습니다. \\ 
    항상 의문을 제기하고 모호한 부분을 제거하고자 노력하고 있으며, 코드 리팩토링을 즐겨하는 편입니다. \\
    다양한 환경에서 개발을 경험해왔으며, 어떤 프로젝트에 참여하던 주도적으로 기여 하고있습니다.
    }

%-----------EXPERIENCE-----------------
\section{Employment}
  \resumeEmploymentListStart
    \resumeEmployment
      {[NHN]}{경기도 성남시, 대한민국}
      {[백엔드개발]}{2022.12 - 재직 중}
        \resumeItemListStart
            \resumeItem{[슬롯개발팀/슬롯서버파트] (파트장)}
            \resumeItem{[슬롯게임제작AI연구TF팀].}
        \resumeItemListEnd   
        
    \resumeEmployment
      {[더퓨쳐컴퍼니]}{서울시 강남구, 대한민국}
      {[블록체인백엔드개발] (Full-time)}{2022.02 - 2022.11}
      \resumeItemListStart
          \resumeItem{[블록체인백엔드팀].}
      \resumeItemListEnd

    \resumeEmployment
      {[엠씨에스텍]}{전남 나주시, 대한민국}
      {[웹개발] (Freelancer)}{2021.08 - 2022.01}
      \resumeItemListStart
          \resumeItem{[IT개발팀].}
      \resumeItemListEnd

    \resumeEmployment
      {[엠씨에스텍]}{전남 나주시, 대한민국}
      {[웹개발/앱개발]}{2018.08 - 2020.12}
      \resumeItemListStart
          \resumeItem{[IT개발팀].}
      \resumeItemListEnd
      
  \resumeEmploymentListEnd

%-----------Projects-----------------

\section{Projects}
  \resumeSubHeadingListStart
    \resumeProject
      {[페블시티 소셜카지노 게임 백엔드 개발]}
         {[앱으로 동작하는 소셜카지노 게임 페블시티의 백엔드를 개발 합니다].}
         \resumeItemListStart
             \resumeItem{[사용 기술 : Java, Spring, MySQL, Redis, Kafka, RabbitMQ, Jenkins, AWS] \\
             \textbf{Key achievements}: \\ 
             1. 저를 포함한 팀원 4명의 파트장을 맡고 있으며, 슬롯게임 서버 파트를 이끌고 있습니다. \\
             2. 슬롯 게임 로직 개발 및 슬롯 서버 전반적인 코드를 책임지고 담당합니다. \\
             3. 반복되는 슬롯 게임 로직들을 템플릿화하여 개발 효율화를 진행하고 있습니다. \\
             4. 슬롯 게임의 RTP(Return of Pay)를 계산하기 위한 시뮬레이터를 유지보수 하고 있습니다. \\
             5. 테스트 코드 작성을 주도하여 안정적인 게임을 만들 수 있도록 기여하고 있습니다.
             6. 생산성 향상을 위해 AI 도구를 적극적으로 도입하여 생산성을 높이고 있습니다. \\}
         \resumeItemListEnd
    
    \resumeProject
      {[바일로스포츠 블록체인 스포츠베팅 게임 웹 풀스택 개발]}
          {[웹으로 동작하는 스포츠베팅 게임 바일로스포츠의 백엔드 및 웹페이지를 개발 했습니다.]}
          \resumeItemListStart
            \resumeItem{[사용 기술 : Java, Spring, svelte, MySQL, RabbitMQ, Jenkins, Azure]} \\
            \textbf{Key achievements}: \\ 
                 1. 늦게 팀원으로 합류하여, 주도적으로 어드민 제작 및 다양한 업무 효율화에 기여했습니다. \\
                 2. 기존 웹 프로젝트에서 사용하던 구조를 점진적으로 개선했습니다.
                 svelte + rollup \longrightarrow svelte + vite \longrightarrow sveltekit \\
        \resumeItemListEnd

    \resumeProject
      {[메타버스2 블록체인 백엔드 개발]}
          {[웹으로 동작하는 메타버스 게임인 메타버스2의 블록체인 관련 백엔드를 개발 했습니다.]}
          \resumeItemListStart
            \resumeItem{[사용기술 : Typescript, NestJS, Web3.js, Metaplex, PostgreSQL, Redis, AWS, Docker]} \\
            \textbf{Key achievements}: \\ 
                 1. 증권거래소와 비슷한 방식의 아이템 거래소 백엔드를 개발했습니다. \\
                 2. 다양한 코인 및 토큰을 입/출금할 수 있는 시스템의 백엔드를 개발했습니다. \\
        \resumeItemListEnd
  \resumeSubHeadingListEnd

%-----------Skills-----------------

\section{Skills}
  \resumeSubHeadingListStart
    \resumeSkills{\textbf{Programming} - [Java], [Typescript]}
    \resumeSkills{\textbf{Frameworks} - [Spring Framework], [NestJS], [Next.js]}
    \resumeSkills{\textbf{[Some skills]} - [AI MCP]}
  \resumeSubHeadingListEnd

%-----------EDUCATION-----------------
\section{Education}
  \resumeSubHeadingListStart
    \resumeSubheading
      {[전남대학교]}{광주광역시, 대한민국}
      {[이학사] [학점 3.18/4.5]}{2012.03 -- 2019.02}
      \resumeItemListStart
          \resumeItem{[자연과학대학 수학과 전공], [공과대학 전자컴퓨터공학부 소프트웨어공학 부전공]}
      \resumeItemListEnd

  \resumeSubHeadingListEnd

%-----------Community-----------------
\section{Side Work}
    \resumeSubHeadingListStart
        \resumeCommunity{[\href{http://www.highturbo.co.kr}{하이터보 회사 웹페이지 제작}]}{2024.12 -- 2025.02}
        {[사용기술 : Next.js, TailwindCSS, Supabase, Vercel]}
        
        \resumeCommunity{[\href{http://www.klea.re.kr}{한국지방교육행정연구재단 웹페이지 제작}]}{2024.07 -- 2024.08}
        {[사용기술 : Next.js, TailwindCSS, Supabase, Vercel]}
    
    \resumeSubHeadingListEnd

%-------------------------------------------
\end{document}
